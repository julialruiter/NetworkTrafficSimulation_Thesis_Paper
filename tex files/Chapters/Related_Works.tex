\chapter{Existing Traffic Simulation Models}
\label{Related}

\par The majority of popular open-source network modeling software focus explicitly on a single network type.  The two most widely used tools and models are SUMO, which specializes in tackling urban traffic simulations, and NS2, which specializes in communication networks. //

SUMO has become the de-facto simulator used to research mobility problems (IEEE eXplore notes nearly 500 citations for the program's inaugural conference paper), allowing researchers to plan and execute urban planning schemes that would be too costly (or time consuming, or flat-out impossible) to enact \cite{LWB18}.  SUMO works by giving its users a graphical user interface where they can import geospatial maps data (typically from Open Street Maps, but users can also draw a graph) and overlay traffic events.  In the GUI, users can add lanes to existing roads or change the type of vehicles allowed per lane (car, bus, high-occupancy vehicle, bike, etc), adding edges to the network and adding (or removing) restrictions to those edges \cite{SUMO2022}. While heavy emphasis is placed on using the GUI, users can also opt to import files and run simulations using Python.\\

Once the road network setup is complete, the user can move on to the "Demand" setup, which is how SUMO describes adding vehicular traffic.  Through the GUI (or importing xml files), users can add vehicles, specifying numerous vehicle attributes like vehicle type and route information\cite{SUMO2022}.  The car creates a \textit{demand} on the network to enter and exit at a particular timestamp, then the network simulation tries to execute it. \\ 

While these features are what make SUMO excel at urban mobility simulation, because of its dependence on geographic location, its use cases cannot be abstracted out \cite{LWB18}. This means that even similar network problems like pedestrian mobility on the same network, simulating truckers who communicate with one another on the same network, or allowing an individual to move from one type of transportation to another require the additional use-case-specific packages and simulators to run on top \cite{NAHRI2021469}, \cite{LWB18}, \cite{SKMR14}. The intersection of traffic and mobility (as in the aforementioned adaptations to SUMO) has resulted in an entire genre of simulators called VANET (Vehicular Ad-Hoc Network).  However, VANET simulations cannot be evaluated directly by SUMO nor the appended packages, thus requiring yet another set of simulation packages to run on top of SUMO \cite{LC08}.\\ 

\par Of course, the same network traffic systems outlined above can be modeled outside of the SUMO environment entirely.  Far preceding the creation of SUMO, papers on trucking and transport simulations can be found dating back to the '90s, even using programs like Microsoft Excel to run them \cite{Dag94}. This leads us to question whether SUMO's painstakingly created attributes and dependencies are even necessary for spinoff simulations. \\

\par To see an example of a more generalized framework in action, we can look at the other popular network traffic simulation framework:  NS2.  NS2 is a "object oriented simulator" tool which was designed explicitly for the simulation and assessment of computer communication networks.  The user manual "Introduction to Network Simulator NS2" details how to use the system for an extensive variety of communication-based simulations it supports have grown extensively in the 30+ years of its existence \cite{IH11}.  Because it is a solid base for simulations where objects may talk with one another, it has seen use beyond its original scope (including many uses in conjunction with SUMO \cite{LWB18}). \\

\par While SUMO allows users to intuitively approach simulation from a visual perspective, NS2 does not, nor should it:  computer communication is somewhat abstract and very intangible, so users to not expect a GUI for setup (though users can then run the set up simulation in a viewing window).  Instead NS2 presents itself as a library users can install and use with Python and runs on C++ objects.  NS2 further differs in that is enables the use of distributed computing to run the traffic events \cite{IH11}.  Rather than modeling location of an object on an edge directly, NS2 operates by calculating the time delay between nodes, and extrapolating location based on that. This corroborates NS2's claim to be a purely mathematical model \cite{IH11}.  \\

\par The design of this traffic simulator has drawn heavily on SUMO for guidance in accommodating "vehicle" attributes; this allows for nuanced simulations to run and supports full customizaibility from the user in how and when the "vehicle" objects can make changes to their route.  But diverging from SUMO and borrowing from NS2, this traffic simulator has tried to keep the simulation as mathematical and modular as possible, ensuring that the components can run in any order and regardless of user interference.  Taking inspiration from the detail of SUMO and the flexibility of NS2, this project aims to create a self-contained simulation environment that can handle a range of use-cases (provided the user has adequate configuration data available).  