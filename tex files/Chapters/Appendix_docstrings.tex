\chapter{Appendix}
\label{Appendix}

\par This sections contains excerpts from the Network Traffic Simulator code and an overview of the system this software was created and tested on.  The version you see here is the complete documentation for version 1.0.0, released  1 July 2022.  \\

\noindent The features found in this version were created to meet (and exceed) the requirements and expectations set  by Universiteit Utrecht staff and serves as a graduation thesis project for their MSc Applied Data Science program.  This release, while a bit barebones on the generator side, sets up a complete, adaptable framework for urban traffic simulations (and other use cases!). \\

\noindent Please check the project's repository (below) for the latest instantiation:
\begin{verbatim}
    https://github.com/julialruiter/Traffic_Simulator
\end{verbatim} 

%%%%%%%%%%%%%%%%%%%%%%%%%%%%%%%%%%%%%%%%%%%%%%%%%%%%%%%%%%%%%%%%%
%---------------------------------------------------------------%
\section{Dependencies}
So much as possible, libraries have been kept to the standard Python libraries.  This means that the current Network Traffic Simulator software version has no external dependencies.

%%%%%%%%%%%%%%%%%%%%%%%%%%%%%%%%%%%%%%%%%%%%%%%%%%%%%%%%%%%%%%%%%
%---------------------------------------------------------------%
\section{System Specifications}
This software was created and tested on a machine with the following specifications.  Though no official stress-testing has been done for this publication, this information may be relevant in analyzing any metrics reported in follow ups: \\

\begin{center}
    \noindent \begin{tabular}{l l}  %{>{\bfseries}l p{6cm}}
    \toprule
    \textbf{OS Name} & Microsoft Windows 10 Home\\
    \midrule
    \textbf{OS Version} & 10.0.19044 N/A Build 19044\\
    \midrule
    \textbf{OS Configuration} & Standalone Workstation\\
    \midrule
    \textbf{OS Build Type} & Multiprocessor Free\\
    \midrule
    \textbf{System Manufacturer} & ASUSTeK COMPUTER INC.\\
    \midrule
    \textbf{System Model} & VivoBook\_ASUSLaptop X521EA\_S533EA\\
    \midrule
    \textbf{System Type} &  x64-based PC\\
    \midrule
    \textbf{Processor} & Intel64 Family 6 Model 140 \\
    & Stepping 1 GenuineIntel ~2803 Mhz \\
    \midrule
    \textbf{Installed RAM} & 16.0 GB (15.7 GB usable)\\
    \bottomrule
    \end{tabular}
\end{center}

 
%%%%%%%%%%%%%%%%%%%%%%%%%%%%%%%%%%%%%%%%%%%%%%%%%%%%%%%%%%%%%%%%%
%---------------------------------------------------------------%
\section{Example Config Files}
\label{Configs}
This section contains the full expected config file structure referenced in\autoref{Manual}, \textit{"\nameref{Manual}"}.  Mandatory fields for the configs have been bolded:

\subsection{Car object format}

\begin{alltt}
\{
    "car_list": [
        \{
            "\textbf{id}": <int>,
            "\textbf{start_edge}": <int>, 
            "start_pos_meter": <float>, 
            "\textbf{end_edge}": <int>,
            "end_pos_meter": <float>,
            "path": <list of consecutive edge ids>,
            "car_length": <float>, 
            "car_type": <"Static" | "Dynamic">,
            "route_preference": <"Shortest" | "Fastest" | 
                                 "Random">,
            "max_tick_potential": <0 < float ≤ 1>
        \},
        ...
    ]
\}
\end{alltt}


\subsection{Network object format}

\begin{alltt}
\{
    "node_list": [
        \{
            "\textbf{id}": <int>,
            "\textbf{intersection_time_cost}": <int>,
            "stoplight_pattern": <list of int lists>,
            "stoplight_duration": <int>,
            "stoplight_delay": <int>
        \},
        ...
    ],
    "edge_list": [
        \{
            "\textbf{id}": <int>,
            "\textbf{start_node_id}": <int>,
            "\textbf{end_node_id}": <int>,
            "edge_length": <float>,
            "max_speed": <float>,
            "max_capacity": <int | Infinity>
        \},
        ...
    ]
\}
\end{alltt}


\section{Example Simulator Setup Code}
\label{SampleCode}

\par To translate the Simulation example from \autoref{Manual}, \textit{"\nameref{Manual}"} into code, we may write the following:

\begin{alltt}
from \textbf{Traffic} import \textbf{TrafficManager}
\textbf{import} json

if __name__ == "__main__":

    # import configs
    network_config_original = None
    try:
        with open("./configs/road_data.json") as 
                  original_json_file: 
            network_config_original = json.load(
                                      original_json_file)
    except Exception as E:
        print(E)
        
    network_config_newlane = None
    try:
        with open("./configs/road_data_extralane.json") as 
                  newlane_json_file: 
            network_config_newlane = json.load(newlane_json_file)
    except Exception as E:
        print(E)

    car_config = None
    try:
        with open("./configs/generated_cars.json") as car_file:  
            car_config = json.load(car_file)
    except Exception as E:
        print(E)
        
    # batch cars
    cars_batch_1 = car_config["car_list"][0:400]
    cars_batch_2 = car_config["car_list"][400:1200]
    # ...etc
    
    # set up the 2 simulations
    tm_original = TrafficManager(network_config_original)
    tm_newlane = TrafficManager(network_config_newlane)
    
    # advance timestamps for simulation 1, adding cars when 
          necessary, outputting snapshots when necessary
    for car in cars_batch_1:
        tm_original.add_car(car)
    for tick in range(1800):     # 30 min
        tm_original.tick()
        if tick \% 60 == 0:
            with open(str(tm_original.get_timestamp()) + 
                          '_snapshot.json', 'w') as f:
                json.dump(tm_original.get_snapshot(), f)
    
    # ...etc for remaining car and time batches
    #  repeat "advance timestamps" for tm_newlane
\end{alltt}

