\chapter{Appendix}
\label{Appendix}

\par This sections contains excerpts from the Network Traffic Simulator code and an overview of the system this software was created and tested on.  The version you see here is the complete documentation for version 1.0.0, released  1 July 2022.  \\

\noindent The features found in this version were created to meet (and exceed) the requirements and expectations set  by Universiteit Utrecht staff and serves as a graduation thesis project for their MSc Applied Data Science program.  This release, while a bit barebones on the generator side, sets up a complete, adaptable framework for urban traffic simulations (and other use cases!). \\

\noindent Please check the repository below for the latest instantiation:
\begin{verbatim}
    https://github.com/julialruiter/Traffic_Simulator
\end{verbatim} 

%%%%%%%%%%%%%%%%%%%%%%%%%%%%%%%%%%%%%%%%%%%%%%%%%%%%%%%%%%%%%%%%%
%---------------------------------------------------------------%
\section{Dependencies}
So much as possible, libraries have been kept to the standard Python libraries.  This means that the current Network Traffic Simulator software version has no external dependencies.

%%%%%%%%%%%%%%%%%%%%%%%%%%%%%%%%%%%%%%%%%%%%%%%%%%%%%%%%%%%%%%%%%
%---------------------------------------------------------------%
\section{System Specifications}
This software was created and tested on a machine with the following specifications.  Though no official stress-testing has been done for this publication, this information may be relevant in analyzing any metrics reported in follow ups: \\

\begin{center}
    \noindent \begin{tabular}{l l}  %{>{\bfseries}l p{6cm}}
    \toprule
    \textbf{OS Name} & Microsoft Windows 10 Home\\
    \midrule
    \textbf{OS Version} & 10.0.19044 N/A Build 19044\\
    \midrule
    \textbf{OS Configuration} & Standalone Workstation\\
    \midrule
    \textbf{OS Build Type} & Multiprocessor Free\\
    \midrule
    \textbf{OS Build Type} & Multiprocessor Free\\
    \midrule
    \textbf{System Manufacturer} & ASUSTeK COMPUTER INC.\\
    \midrule
    \textbf{System Model} & VivoBook\_ASUSLaptop X521EA\_S533EA\\
    \midrule
    \textbf{System Type} &  x64-based PC\\
    \midrule
    \textbf{Processor} & Intel64 Family 6 Model 140 \\
    & Stepping 1 GenuineIntel ~2803 Mhz \\
    \midrule
    \textbf{Installed RAM} & 16.0 GB (15.7 GB usable)\\
    \bottomrule
    \end{tabular}
\end{center}
 \\
 
%%%%%%%%%%%%%%%%%%%%%%%%%%%%%%%%%%%%%%%%%%%%%%%%%%%%%%%%%%%%%%%%%
%---------------------------------------------------------------%
\section{API Module: Traffic.py}
This module instantiates a simulation instance and exposes all necessary/desired functions for interacting with the simulation once it is running:

\subsection{class TrafficManager:}

\begin{alltt}
\textbf{__init__}(self, network_config) -> None
Establishes an instance of TrafficManager to run on the given 
network structure.
Attributes:
    graph:  Network object that the TrafficManager runs on.
    timestamp:  Simulation timestamp.

\textbf{add_car}(self, car)
API function:  place car (dictionary object) onto the network's 
waiting queue.
Please note that the current iteration of add_car only supports 
placement directly onto Edges (identified by "start_edge").

\textbf{get_all_paths_A_to_B}(self, start_edge_ID, end_edge_ID)
API function:  Given a start and end Edge id, return a list of 
all valid paths that do not repeat Edges.

\textbf{get_node_edges_in_out}(self, node_ID)
API function:  lists the IDs of inbound and outbound edges for 
a particular node.
This information can also be found via 'get_snapshot()'.

\textbf{get_path_distance}(self, path)
API function:  Given the ordered list of Edges as "path", 
evaluate the total distance it would take to travel.
This function assumes that the entirety of each Edge is 
traveled.

\textbf{get_path_minimum_time}(self, path)
API function:  Given the ordered list of Edges as "path", 
evaluate the minimum time it would take to travel (in ticks) 
given each Edge's speed limit.
Minimum time is calculated assuming a car is able to travel 
the maximum speed per edge unencumbered.
This function assumes that the entirety of each Edge is 
traveled and includes any Node-crossing time penalties.
\textit{Note:  time cost does NOT include Node-crossing time out of the 
final edge as the Car is expected to exit the Network before the 
Edge's end.}

\textbf{get_shortest_path_A_to_B}(self, all_paths_list)
API function:  Given all_paths_list (a list of paths from A to 
B as calculated using self.get_all_paths_A_to_B()),
returns the path with the shortest total distance in terms of 
length.

\textbf{get_snapshot}(self)
API function:  outputs list of nodes, edge attributes, car 
attributes.
Output is formatted in such a way that it can be used as input 
for a new simulation.

\textbf{get_theoretical_fastest_path_A_to_B}(self, all_paths_list)
API function:  Given all_paths_list (a list of paths from A to 
B as calculated using self.get_all_paths_A_to_B()),
returns the path with the minimum total travel time (assuming no
congestion).

\textbf{get_timestamp}(self)
API function:  returns (sequential) state number.

\textbf{pause_car}(self, car_id)
API function:  forcefully halts the Car associated with 'car_id' 
until a 'resume_car' call is received.
No cars behind it may pass.

\textbf{remove_car}(self, car_id)
API function:  removes the Car associated with 'car_id' from
the simulation.
This is done by forcing it into Car.status = 'Removed from 
simulation'.

\textbf{resume_car}(self, car_id)
API function:  allows the Car associated with 'car_id' to 
resume moving.

\textbf{tick}(self)
API function:  advance state of network by one unit of time.


\end{alltt}

%%%%%%%%%%%%%%%%%%%%%%%%%%%%%%%%%%%%%%%%%%%%%%%%%%%%%%%%%%%%%%%%%
%---------------------------------------------------------------%
\section{Hidden Simulation Module: traffic\_network.py}

This module creates and runs an instance of a Network object when a simulation is created via starting a \textbf{TrafficManager} instance.  The main item here is \textbf{class Network}, which  contains, creates, and collects two dependent components (\textbf{class Node} and \textbf{class Edge}.  \\

\noindent The functions here should be hidden from the user.  Instead, the user should call for changes using the\textbf{TrafficManager} API.  Internal functions belonging to the \textbf{traffic\_network} module can be seen below:

%%%%%%%%%%%%%%%%%%%%%%%%%%%%%%%%%%%%%%%%%%%%%%%%%%%%%%%%%%%%%%%%%
%---------------------------------------------------------------%
\subsection{class Network:}

\begin{alltt}
\textbf{__init__}(self, TrafficManagerPointer, config) -> None
Contains all functions and attributes pertaining to the (road) 
network as a whole.
Attributes:
    \textbf{TrafficManager_pointer}:  Identifies which TrafficManger 
    simulation is associated with this network
    \textbf{node_ID_to_node}:  Dictionary mapping Node IDs to Node objects.
    \textbf{edge_ID_to_edge}:  Dictionary mapping Edge IDs to Edge objects.
    \textbf{car_ID_to_car}:  Dictionary mapping Car IDs to Car objects.
    \textbf{global_tick}:  Tick index, aligns with TrafficManager tick

\textbf{add_car}(self, car)
Places Car (object) on the waiting queue for its specified 
start_edge.
Validity checks have been passed up to the TrafficManager level 
as a part of "add_car(car)",
ensuring that any Cars received are valid OR can be made valid 
using the DEFAULT_car_values_config.json file.

\textbf{add_edge}(self, edge)
Imports edge(s) from given edge dictionary.
If both the start and end nodes are already in the network, then 
the edge will be added.
Any Edge attribute values not given in the edge object (imported) 
will instead be assigned from the imported defaults file:
edge_default_config.
\textit{Note:  there are no default values for id, start_node_id, nor 
end_node_id as these are an Edge's unique identifiers.}

\textbf{add_node}(self, node)
Imports node(s) from given node dictionary and adds them to 
the network.

\textbf{all_paths_depth_first_search}(self, current_edge_ID, 
end_edge_ID, visited_list=[], valid_paths=[])
Given a start and end Edge id, return a list of all valid paths 
that do not repeat Edges.

\textbf{check_valid_car}(self, car)
Returns a detailed Exception if the given car does not conform 
to expected input structure.

\textbf{choose_path}(self, all_paths_list, metric)
Given a list of paths from A to B (ex: as calculated using
self.all_paths_depth_first_search()),
returns the "best" path with regards to input metric.
Currently supported input metrics:
    \textbf{'Fastest'}: best path = minimum total travel time (assuming 
    no congestion).
    \textbf{'Shortest'}: best path = shortest total distance in terms of 
    length.
    \textbf{'Random'}:  pay no heed to metics, choose an available path 
    at random.
Future versions may include metrics like:
    \textbf{'Fastest_now'}: best path = minimum travel time after 
    accounting for current Network congestion.

\textbf{get_Network_pointer}(self)
Returns tick index (which aligns with TrafficManager tick).

\textbf{get_edge_id}(self, edge_id)
Uses Network.edge_ID_to_edge dictionary to map an Edge ID to 
its corresponding Edge object.

\textbf{get_global_tick}(self)
Returns a pointer to this Network instance.
Useful for directly generating cars OR keeping track managing 
multiple simulations at once.

\textbf{get_node_from_id}(self, node_id)
Uses Network.node_ID_to_node dictionary to map a Node IDs to 
its corresponding Node object.

\textbf{get_snapshot}(self)
Outputs dictionary containing snapshot data for all nodes and 
edges in the network.

\textbf{path_cost_distance}(self, path_list)
Given path_list, evaluate the total distance it would take to 
travel.
This function assumes that the entirety of each Edge is traveled.

\textbf{path_cost_minimum_time}(self, path_list)
Given path_list, evaluate the the minimum time it would take to 
travel (in ticks) given each Edge's max_speed.
Minimum time is calculated assuming a car is able to travel the 
maximum speed per edge unencumbered.
This function assumes that the entirety of each Edge is traveled 
and includes any Node-crossing time penalties.
\textit{Note:  time cost does NOT include Node-crossing time out of the 
final edge as the Car is expected to exit the Network before the 
Edge's end.}

\textbf{remove_edge}(self, edge)
Placeholder for future software version:
Will remove an Edge and all of its associated Cars from the 
Network.

\textbf{remove_node}(self, node)
Placeholder for future software version:
Will remove a Node and all of its associated inbound/outbound 
Edges from the Network.

\textbf{restore_tick_potential}(self)
Resets the tick_potential to its maximum value for all Cars on 
the Network.

\textbf{tick}(self)
Shuffles the order in which Node ticks will be processed with 
each global tick to ensure no node is favored.
\textit{Note:  global tick != Node tick.  Global tick is the unit of 
time until the next state of the simulation, 
while Node tick the proportion of that time that its components 
can move uninterrupted.  
Node ticks will occur until the sum of their durations reaches 
that of a global tick/no further movement is possible.}

\end{alltt}


%%%%%%%%%%%%%%%%%%%%%%%%%%%%%%%%%%%%%%%%%%%%%%%%%%%%%%%%%%%%%%%%%
%---------------------------------------------------------------%
\subsection{class Node:}

\begin{alltt}
\textbf{__init__}(self, Network_reference, id, intersection_cost,
stoplight_pattern, stoplight_duration, stoplight_delay) -> None
Contains all functions and attributes pertaining to a network 
intersection (Node).
Attributes:
    \textbf{id}:  Unique ID associated with this Node object.
    \textbf{inbound_edge_ID_to_edge}:  Dictionary mapping inbound Edge 
    IDs to Edge objects.
    \textbf{outbound_edge_ID_to_edge}:  Dictionary mapping outbound Edge 
    IDs to Edge objects.
    \textbf{intersection_time_cost}:  Value representing time in ticks 
    required to cross intersection.  0 <= value < 1.
    \textbf{stoplight_pattern}:  Ordered list of sets of simeltaneous 
    Edges eligible for car exiting. Pattern cycles through sets. 
    (Will be implemented in future versions of the software).
    \textbf{stoplight_pattern_current_index}:  Index representing which 
    set of stoplight_pattern the Node is currently on.  (Will be
    implemented in future versions of the software).
    \textbf{stoplight_duration}: Number of ticks that the stoplight_pattern 
    stays on its current Edge set. (Will be implemented in future 
    versions of the software).
    \textbf{stoplight_delay}: Number of ticks between change of 
    stoplight_pattern Edge sets. (Will be implemented in future 
    versions of the software).
    \textbf{node_tick_number}:  Used in stoplight changes, increments by 
    one with each global TrafficManager tick. 

\textbf{add_to_inbound}(self, edge)
Used when adding an Edge to the Network when 
Edge.end_node == self.id .
Adds the Edge ID and a mapping to its corresponding Edge object 
to the inbound_edge_ID_to_edge dictionary.

\textbf{add_to_outbound}(self, edge)
Used when adding an Edge to the Network when 
Edge.start_node == self.id .
Adds the Edge ID and a mapping to its corresponding Edge object 
to the outbound_edge_ID_to_edge dictionary.

\textbf{get_inbound_exit_candidates}(self)
Checks all inbound edges of a Node.  
Any edge that has a Car at the end position of its length is 
considered a candidate to advance on to the next Edge in its 
path.

\textbf{get_intersection_time_cost}(self)
Returns self.intersection_time_cost, the time penalty it takes 
to cross a Node.
\textit{Note:  this value may be 0.}

\textbf{get_node_ID}(self)
Returns self.id.
Used when calling value from outside the Node class.

\textbf{get_node_inbound}(self)
Returns the keys to the dictionary self.inbound_edge_ID_to_edge, 
list of all inbound Edge IDs.
Used when calling from outside the Node class.

\textbf{get_node_outbound}(self)
Returns the keys to the dictionary self.outbound_edge_ID_to_edge, 
list of all outbound Edge IDs.
Used when calling from outside the Node class.

\textbf{get_snapshot}(self)
Outputs dictionary of Node attributes.

\textbf{tick}(self)
Facilitates Edge ticks and movement of Car objects from one Edge 
to another.
If a Car that is eligible to cross the Node has type "Dynamic", 
then its path is recalculated upon crossing.
Each Node tick shuffles the order in which Edges tick to ensure 
no particular Edge is favored.

\textbf{update_stoplight_attributes}(self)
Toggles which Edges allow cars to exit by cycling through sets 
in stoplight_pattern.
Returns the set of which inbound Edge set in stoplight_pattern is
currently active, or NULL set (denoting red lights for all edges).
Communication to TrafficManager to establish global tick and 
increment will be solved in future versions.
Communication to prevent cars from leaving inbound Edges during 
the node tick will be established in future versions.

\end{alltt}

%%%%%%%%%%%%%%%%%%%%%%%%%%%%%%%%%%%%%%%%%%%%%%%%%%%%%%%%%%%%%%%%%
%---------------------------------------------------------------%
\subsection{class Edge:}

\begin{alltt}
\textbf{__init__}(self, id, start_node_id, end_node_id, edge_length,
max_speed, max_capacity) -> None
Contains all functions and attributes pertaining to a road 
segment (Edge).
Attributes:
    \textbf{id}:  Unique ID associated with this Edge object.
    \textbf{start_node_id}:  Node from which this Edge originates (this
    Edge is an outbound_edge for start_node).
    \textbf{end_node_id}:  Node from which this Edge terminates (this 
    Edge is an inbound_edge for end_node).
    \textbf{start_node}:  Node object represented by start_node_id.
    \textbf{end_node}:  Node object represented by end_node_id.
    \textbf{edge_length}:  Physical length of the Edge (ex: meter length
    of a road).  Default value can be found and adjusted at
    edge_default_config["edge_length"]
    \textbf{max_speed}:  (optional) Unit speed limit of the road.  
    Without obstructions, this is the maximum distance a Car 
    can move on this Edge in one tick.  Default value can be 
    found and adjusted at edge_default_config["max_speed"]
    \textbf{max_capacity}:  (optional) Maximum number of Car objects
    allowed on the Edge (max length of current_cars). Default 
    value can be found and adjusted at 
    edge_default_config["max_capacity"]
    \textbf{edge_car_ID_to_car}:  Dictionary containing all Car objects
    associated with the Edge; maps Car IDs to Car objects.
    \textbf{current_cars}:  List of IDs of all Cars currently on the Edge.
    \textbf{waiting_cars}:  List of IDs for Cars that are trying to enter
    the Network at this Edge.
    \textbf{processed_cars}:  List capturing IDs of Cars that have already
    been processed on the current tick.  Becomes current_cars at 
    the end of the Edge tick.
    \textbf{completed_cars}:  List of IDs of any Cars that have completed
    their route on this Edge in the duration of the simulation.
\textit{Note:  some attributes have been given default values in the case
that the user did not provide them.}

\textbf{add_car_to_wait_queue}(self, car)
Adds Car object to the waiting queue and links Car to Edge on Car 
ID.

\textbf{get_current_cars}(self)
Returns self.current_cars, the list of Car IDs for all cars 
currently on the Edge.
Used when calling value from outside the Edge class.

\textbf{get_edge_ID}(self)
Returns self.id.
Used when calling value from outside the Edge class.

\textbf{get_end_node}(self)
Returns self.end_node Object.
Used when calling value from outside the Edge class.

\textbf{get_end_node_id}(self)
Returns self.end_node_id.
Used when calling value from outside the Edge class.

\textbf{get_length}(self)
Returns self.edge_length (length of road segment, typically in 
meters).
Used when calling value from outside the Edge class.

\textbf{get_max_capacity}(self)
Returns self.max_capacity.
Used when calling value from outside the Edge class.

\textbf{get_max_speed}(self)
Returns self.max_speed (speed limit of road segment, typically
in meters/sec).
Used when calling value from outside the Edge class.

\textbf{get_snapshot}(self)
Outputs dictionary of Edge attributes, including lists of Cars 
that are:  currently on the Edge, waiting to enter the Edge, or 
completed their trip on this Edge.

\textbf{get_start_node}(self)
Returns self.start_node Object.
Used when calling value from outside the Edge class.

\textbf{get_start_node_id}(self)
Returns self.start_node_id.
Used when calling value from outside the Edge class.

\textbf{move_existing_car_to_edge}(self, car)
Adds Car object to the 'processed-cars' list and links Car to 
(new) Edge on Car ID.

\textbf{set_current_cars}(self, new_list)
Replaces the list contents of self.current_cars with new_list.
Used when updating value from outside the Edge class.

\textbf{set_end_node}(self, node_ptr)
Associates (end) Node pointer with Edge object.
Used when adding an Edge to the Network.

\textbf{set_start_node}(self, node_ptr)
Associates (start) Node pointer with Edge object.
Used when adding an Edge to the Network.

\textbf{tick}(self)
Facilitates the movement of Car objects traversing this Edge.  
There are three types of movement:
\textbf{car entry}:  a Car from the waiting_car list will be placed on 
the Edge if and when space becomes available.
\textbf{car exiting}:  a Car will exit the Network if and when it reaches
its end_pos_meter in the process of its movement IF 
self.id = Car.end_edge.
\textbf{car movement}:  a Car with status mobile = True will advance as
far as possible (maximum potential distance, edge end, or until 
obstructed by another car).

\end{alltt}


%%%%%%%%%%%%%%%%%%%%%%%%%%%%%%%%%%%%%%%%%%%%%%%%%%%%%%%%%%%%%%%%%
%---------------------------------------------------------------%
\section{Hidden Simulation Module: network\_cars.py}

\par This module creates and stores a car object.  A car is created when called into existence by an API call via the \textbf{Traffic} module.  While the \textbf{network\_cars} module is fully dependent on the \textbf{network\_traffic} module to move, car objects can exist separately.  Thus,  \textbf{network\_cars} is imported into the \textbf{network\_traffic} module to allow for object-network interaction. \\

\noindent Once again, the functions here should be hidden from the user.  Instead, the user should call for changes and additions using the\textbf{TrafficManager} API.  Internal functions belonging to the \textbf{network\_traffic} module can be seen below:

%%%%%%%%%%%%%%%%%%%%%%%%%%%%%%%%%%%%%%%%%%%%%%%%%%%%%%%%%%%%%%%%%
%---------------------------------------------------------------%

\subsection{class Car:}

\begin{alltt}
\textbf{__init__}(self, car_ID, car_length, start_edge, start_pos_meter,
end_edge, end_pos_meter, path, car_type, route_preference,
max_tick_potential) -> None
Contains all functions and attributes pertaining to an object 
traversing the Network (Car).
Attributes:
    \textbf{id}:  Unique ID associated with this Car object.
    \textbf{car_length}:  Physical unit length of the Car object 
    (ex: meters).  May be 0.
    \textbf{start_edge}:  Edge from which this Car originates 
    its journey.
    \textbf{start_pos_meter}:  Unit position along start_edge 
    from which the Car begins its journey.  
    Edge origin = position 0.
    \textbf{end_edge}:  Edge from which this Car terminates its 
    journey.
    \textbf{end_pos_meter}:  Unit position along end_edge at 
    which the Car terminates its journey and leaves the Network.
    \textbf{path}:  Ordered list of Edges that the Car will 
    traverse to get from start to end.
    \textbf{route_preference}:  Classification determining 
    which type of path will be followed:
        if \textbf{'Fastest'}: chooses path with minimum total 
        travel time (assuming no congestion).
        if \textbf{'Shortest'}: chooses path with shortest total 
        distance in terms of length.
        if \textbf{'Random'}:  pays no heed to metrics and 
        instead chooses an available path at random.
    \textbf{car_type}:  Car classification for path-following:
        if \textbf{'Static'}:  Car follows predetermined path 
        set on addition.
        if \textbf{'Dynamic'}:  Car will recalculate its route 
        every time it reaches a Node. 
    \textbf{mobile}:  Car classification for mobility:
        if \textbf{True}:  Car is eligible to move (default).
        if \textbf{False}:  Car has been halted and will not move 
        until further instructions given.
    \textbf{route_status}:  string explaining the Car's status 
    with regards to path completion:
        \textbf{'In progress'}:  The Car is eligible for movement; 
        the Car is moving along its path.
        \textbf{'Route Completed'}:  The Car has reached its 
        destination and has been removed from the Network.
        \textbf{'Paused'}:  The Car is ineligible for movement 
        due to mobile=False.
        \textbf{'Removed from simulation at tick #n'}:  The Car 
        was removed from the simulation by external intervention.  
        n denotes timestamp at which it was removed.
    \textbf{current_edge}:  Edge ID corresponding to the Car's 
    current location.
    \textbf{current_pos_meter_car_front}:  Unit distance along
    current_edge corresponding to the Car's current location.  
    If car_length > 0, this refers to the position of the front 
    of the Car.
    \textbf{max_tick_potential}:  Proportion of global maximum 
    tick time-distance that the Car is eligible to move 
    (default = 1, full potential).
    \textbf{current_tick_potential}:  Portion of tick 
    time-distance that the car has not (yet) utilized on this 
    tick.

\textbf{get_car_ID}(self)
Returns self.id.
Used when calling value from outside the Car class.

\textbf{get_car_length}(self)
Returns self.car_length.
Used when calling value from outside the Car class.

\textbf{get_car_type}(self)
Returns self.car_type.
Value is \textbf{"Static"} (Car remains on its original path) or \textbf{"Dynamic"}
(Car recalculates path at every Node crossing).
Used when calling value from outside the Car class.

\textbf{get_current_edge}(self)
Returns self.car_length.
Used when calling value from outside the Car class.

\textbf{get_current_pos_meter_car_front}(self)
Returns self.current_pos_meter_car_front.
Used when calling value from outside the Car class.

\textbf{get_current_tick_potential}(self)
Returns self.current_tick_potential.
Used when calling value from outside the Car class.

\textbf{get_end_edge}(self)
Returns self.end_edge, the Edge at which the Car finishes its 
route and leaves the Network.
Used when calling value from outside the Car class.

\textbf{get_end_pos_meter}(self)
Returns self.end_pos_meter, the position on the Edge at which 
the Car finishes its route and leaves the Network.
Used when calling value from outside the Car class.

\textbf{get_max_tick_potential}(self)
Returns self.max_tick_potential.
Used when calling value from outside the Car class.

\textbf{get_mobility}(self)
Returns self.mobile.
Value is \textbf{True} (Car is eligible to move) or \textbf{False} 
(Car is halted or its path is complete).
Used when calling value from outside the Car class.

\textbf{get_path}(self)
Returns self.path, the ordered list of upcoming Edges the Car
will traverse.
Used when calling value from outside the Car class.

\textbf{get_route_metric}(self)
Returns self.route_preference.  Used when a Car's path needs 
to be (re)calculated.
Value is "Shortest", "Fastest", "Random", with "Random" being 
the default value if none specified.
Used when calling value from outside the Car class.

\textbf{get_route_status}(self)
Returns self.route_status.
Value is "In progress", "Route Completed", "Paused", or 
"Removed from simulation at tick #n".
Used when calling value from outside the Car class.

\textbf{get_snapshot}(self)
Outputs dictionary of Car attributes.

\textbf{get_start_edge}(self)
Returns self.start_edge, the Edge at which the Car entered 
the Network.
Used when calling value from outside the Car class.

\textbf{get_start_pos_meter}(self)
Returns self.start_pos_meter, the position on the start Edge 
at which the Car entered the Network.
Used when calling value from outside the Car class.

\textbf{set_current_edge}(self, edge_ID)
Replaces self.current_edge with edge_ID.
Used when updating value from outside the Car class.

\textbf{set_current_pos_meter_car_front}(self, new_position_meters)
Replaces self.current_pos_meter_car_front with 
new_position_meters.
Used when updating value from outside the Car class.

\textbf{set_current_tick_potential}(self, new_potential)
Replaces self.current_tick_potential with new_potential.
Used when updating value from outside the Car class.

\textbf{set_mobility}(self, Boolean)
Updates the Boolean value of self.mobile to input Boolean.
Value is \textbf{True} (Car is eligible to move) or \textbf{False}
(Car is halted or its path is complete).
Used when updating value from outside the Car class.

\textbf{set_path}(self, new_path_list)
Replaces self.path with new_path_list, 
typically removing the first entry as the car enters a new Edge, 
or when calculating a new route.
Used when updating value from outside the Car class.

\textbf{set_route_status}(self, new_string)
Updates self.route_status to new_string.
Value should be "In progress", "Route Completed", "Paused", or 
"Removed from simulation at tick #n".
Used when updating value from outside the Car class.

\textbf{tick}(self, old_potential)
Calculates "potential" differential;
This is the portion of a full tick movement completed by the 
Car on this tick.

\end{alltt}


%%%%%%%%%%%%%%%%%%%%%%%%%%%%%%%%%%%%%%%%%%%%%%%%%%%%%%%%%%%%%%%%%
%---------------------------------------------------------------%
\section{Optional Module:  \\ UnderlyingNetworkGenerator.py}

In the absence of a pre-made graph to run the simulation on, the user can elect to use this module to generate a network.  This method of Netowrk generation may be useful for some particular use cases where it is beneficial to create a graph with many Nodes and Edges sharing the same properties.\\

\noindent The current version of this module supports the creation of Erdős-Renyi random graphs OR complete, bidirectional graphs.  This module will be improved later to support other mathematical network types.

\subsection{class NetworkGenerator:}

\begin{alltt}
\textbf{__init__}(self) -> None
Class containing various functions for generation Network objects 
for the simulation to run on.
A Network can also be provided via custom JSON file instead.

\textbf{create_ER_network_default_values}(self, number_nodes,
probability_joining=0.5)
Creates an Erdos Renyi Network based on the given parameters:
A each pair of nodes has a probability_joining (0 < p < 1) of 
being connected in an ER Network.
As this is a directional Network, each pair will be considered 
separately per direction.
This Network uses the following default values:
    probability_joining = 0.5     # can be overwritten by input
    node.intersection_time_cost = 0
Please note that this NetworkGenerator function only generates 
the barebone structures necessary for a Network. 
All additional attributes will be loaded via
"DEFAULT_edge_values_config.json" during the simulation process.

\textbf{generate_complete_bidirectional_network_default_values}(self,
number_nodes)
Generates a complete Network consisting of number_nodes Nodes, 
each connected to every other Node in both directions.
This Network uses the following default value:
    node.intersection_time_cost = 0
Please note that this NetworkGenerator function only generates 
the barebone structures necessary for a Network. 
All additional attributes will be loaded via
"DEFAULT_edge_values_config.json" during the simulation process.
output_Network_dictionary(self, node_dict, edge_dict)
Returns dictionary containing all Node and Edge information for 
the newly generated Network.

\end{alltt}

%%%%%%%%%%%%%%%%%%%%%%%%%%%%%%%%%%%%%%%%%%%%%%%%%%%%%%%%%%%%%%%%%
%---------------------------------------------------------------%
\subsection{class GeneratorNode:}

\begin{alltt}
\textbf{__init__}(self, id) -> None
Contains all attributes necessary for creating a network 
intersection (Node).
Attributes:
    id:  Unique ID associated with this Node object.
    intersection_time_cost:  Value representing time in ticks 
    required to cross intersection.  0 <= value < 1.
\end{alltt}


%-----------------------------------------------------------------%
\subsection{class GeneratorEdge:}

\begin{alltt}
\textbf{__init__}(self, id, start_node_id, end_node_id, edge_length=None,
max_speed=None, max_capacity=None) -> None
Contains all attributes necessary for creating a road segment 
(Edge).
Attributes generated in all NetworkGenerator functions:
    \textbf{id}:  Unique ID associated with this Edge object.
    \textbf{start_node_id}:  Node from which this Edge originates.
    \textbf{end_node_id}:  Node from which this Edge terminates.
Attributes generated only in probabilistic NetworkGenerator 
functions:
    \textbf{edge_length}:  Physical length of the Edge (ex: meter length 
    of a road).
    \textbf{max_speed}:  (optional) Unit speed limit of the road.  
    Without obstructions, this is the maximum distance a Car 
    can move on this Edge in one tick.
    \textbf{max_capacity}:  (optional) Maximum number of Car objects
    allowed on the Edge.
\end{alltt}